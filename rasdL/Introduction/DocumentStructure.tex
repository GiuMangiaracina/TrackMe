\section{Document Structure}

This document is divided into 5 sections:

\begin{itemize}

\item \underline{Introduction} \\
It includes a general overview on the System, its goals and all the information useful to better understand the other sections of the document.

\item \underline{Overall Description} \\
It includes the models of the System: UML, shared phenomena and state charts; the domain assumptions and constraints, the requirements associated to each goal and a reference to the users of the application.

\item \underline{Specific Requirements} \\
This is the main body of the document. It contains: the \textit{external interface requirements}, which include the mockup of the application, both mobile and web; the \textit{functional requirements}, with all the scenarios and the use cases of the main functionalities, with the aim of describing how the application will work; the \textit{performance requirements}; all the design constraints; finally, the software system attributes.

\item \underline{Formal Analysis} \\
This section contains the Alloy model of our System, with proofs of correctness of the more relevant assertions and predicates.

\item \underline{Effort Spent} \\
A brief overview on the effort spent by each component of the group.

\end{itemize}