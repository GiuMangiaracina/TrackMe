\section{Definitions, Acronyms, Abbreviations}
In this part of the RASD Document there are some definitions, acronyms and abbreviations that will be used among the following chapters.
\subsection{Definitions}

\begin{itemize}

\item \textbf{Visitor:} when referring to the \emph{Visitor}, we refer to an actor that has not registered to the TrackMe system yet.

\item \textbf{Registered User:} a user registered to the TrackMe System. All actors must be registered to the system in order to perform any action on it.

\item\textbf{Logged-in User:} a registered user who has logged into the TrackMe System

\item \textbf{User:} when referring to the \emph{User}, we refer to a logged-in user which is not a Third party. 

\item\textbf{Organizer:} referring to a run of the Track4Run service, the logged-in user that has created it

\item \textbf{System /TrackMe System:} the entire system infrastructure, encompassing the Data4Help, AutomatedSOS and Track4Run services.

\item\textbf{Health data:} data collected by user's smart device which refer to his/her health status

\item\textbf{Request:} performed by a Third Party, a request asks the permission to get access to some specific user data.
There are 4 types of requests: 
\begin{itemize}
\item \textit{One-shot User requests} concern the latest available data of a single user
\item \textit{Subscription User requests} provide the Third Party with regular updates of the user data
\item \textit{One-shot Group requests} concern the latest available data of a specific set of users
\item \textit{Subscription Group request} provide the Third Party with regular updates of the group data
\end{itemize}
In a request, the Third Party can specify the type of data it wants to receive, with a choice among different health data (heart rate, blood pressure, step counter, etc.). User requests must be accepted by the user to be performed, while group requests are automatically active if the selected set of individuals is greater than 1000 users.

\item\textbf{Run:} sports event organized with the Track4Run System. Scheduled runs are runs that have been created in the system and will take place in the future, while Live runs are those which are taking place at the moment and can be watched live. 

\item\textbf{Path:} route that the participants of a run will follow during the event. It is selected by the run organizer during the event creation (start point, intermediate points and final point)
\end{itemize}

\subsection{Acronyms}

\begin{itemize}
  \item R.A.S.D.: Requirements Analysis and Specifications Document
  \item A.P.I.: Application Programming Interface 
  \item N.H.S.: National Health Service
  \item S.M.S.: Short Message Service
\end{itemize}

\subsection{Abbreviations}

\begin{itemize}
	\item {[}G k{]}: The k-th goal
    \item {[}D k{]}: The k-th domain assumption
    \item {[}R k{]}: The k-th functional requirement
\end{itemize}
