\section{Definitions, Acronyms, Abbrevations}
In this part of the RASD Document there are some definitions, acronyms and abbreviations that will be used among the following chapters.
\subsection{Definitions}
\begin{itemize}
	\item \textbf{The Software:} when referring to \emph{The Software}, this document refers to the entire system infrastructure, at implementation and design level.

\item \textbf{User:}When referring to \emph{The User}, we refer to a logged- in user, since the software is usable only by a recognized user. 

\item \textbf{Visitor:}When referring to \emph{The Visitor}, we refer to a user not yet registered to the TrackMe system

\item \textbf{Registered User:} A User registered to the TrackMe System. To use the System and performs any action, all actors must be registered. 

\item\textbf{Logged-in User:} A registered user who is logged in the TrackMe System

\item\textbf{Organizator:} Is a logged-in  user who uses the Track4Run system, and decides to create a new run

\item\textbf{TrackMe:}
is the name of our System, that  includes our three main services: Data4Help, AutomatedSos and Track4Run.

\item\textbf{NHS:} stands for National Health Service.
\item\textbf{Health data:} With health data we refer to the health data collected by user's device, among those available or requested:heart Beat, blood Pressure, Step counter, etc. 



	
	\end{itemize}
\subsection{Acronyms}

\begin{itemize}
  \item R.A.S.D: Requirements Analysis and Specifications Document
  \item A.P.I: Application Programming Interface 
\end{itemize}

\subsection{Abbrevations}
These abbreviations will be used both in this document and in the follows documents
\begin{itemize}
	\item {[}G k{]}: The k-th goal
    \item {[}D k{]}: The k-th Domain Assumption
    \item {[}R k{]}: The k-th Functional Requirement
\end{itemize}
