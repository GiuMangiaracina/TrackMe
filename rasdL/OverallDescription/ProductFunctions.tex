\section{Product Functions}

Here we list the main requirements, concerning each goal. \\ \\

\textbf{[G1]The Registered User can access to the services offered by TrackMe System with a single account. } \\ \\
R1)The User accesses to the System only through a mobile application. \\ \\
R2)The User can access all the three services (Data4Help, AutomatedSOS and Track4Run). \\ \\
R3)The Third Party can access only the Data4Help service. \\ \\
R4) The Third Party accesses to the System only through a web application. \\ \\
R5)The User can decide to keep AutomatedSOS's service active or deactivate it arbitrarily. \\ \\

\textbf{[G2]The Registered User can be recognized by providing his/her username and password.} \\ \\
R6)A User  have to specify his/her personal data at registration time: first name, last name, fiscal code, age, city, weight and height. \\ \\
R7)The Third Party have to specify its personal data at registration time: company name, account holder, VAT number, address. \newline \newline

\textbf{Data4Help:} \\

\textbf{[G3.1]Allow a User to manage the accesses to his/her personal data.} \\ \\
R8)The User receives in real time requests on his/her data from Third Parties. \\ \\
R9)The Request can be accepted or denied by the User.\\ \\
R10)The User can see the specific type of data requested from the Third Party. \\ \\ 

\textbf{ [G3.2]Allow a User to visualize his/her actual health parameters and position.} \\ \\
R11)The User must be able to see its position on an interactive map. \\ \\
R12)The position must be updated in real time. \\ \\
R13)The System provides the latest health and location data available. \\ \\

\textbf{[G3.3]Allow a User to visualize his/her past data history.} \\ \\
R14)The System asks the User what kind of data he/she wants to see. \\ \\
R15)The data are presented in the order specified by the User.  \\ \\
R15.1)The System provides two grouping options: time and location. \\ \\
R16)A manual research on its own data history can be performed by the User. \\ \\
R16.1)The User can customize its search specifying the time or location range. \\ \\ 

\textbf{[G3.4]Allow a Third Party to send an authorization request to a User for the access to the his/her data.} \\ \\
R17)The System asks to insert the fiscal code of the specific User. \\ \\
R18)The Third Party is asked to specify the type of data to be requested.\\ \\ 
R19)The System notify the User about the request from the Third Party.\\ \\
R20)The System notify the Third Party as soon as the response to their request is available. \\ \\

\textbf{[G3.4.1]Allow a Third Party to request the latest available data of a User.} \\ \\
R21)If the request is accepted by the User, the System will provide the Third Party with his/her latest available data. \\ \\ 

\textbf{[G3.4.2]Allow a Third Party to request a subscription to the data of a Users.} \\ \\
R22)The Third Party can define the interval between updates for each specific parameter. \\ \\
R23)If the request is accepted by the user, the system will provide the third party with periodic updates. \\ \\ \\

\textbf{[G3.5]Allow a Third Party to request anonymized data of a set of Users.} \\ \\
R24) The Third Party is asked to specify the type of data to be requested. \\ \\
R25)The data requested will be chosen among those that match the preferences specified by the Third Party: age range, geographical area, localization, gender, weight, height. \\ \\
R26)The data will be anonymized to prevent the possibility of a misuse of data. \\ \\ 

\textbf{[G3.5.1]Allow the Third party to request the latest data of the set of Users.} \\ \\
R27)If the request is allowed, the System will provide the Third party with its latest available data about group. \\ \\

\textbf{[G3.5.2] Allow a Third Party to request a subscription to the data of the set of Users.} \\ \\
R28)The Third Party can define the interval between updates for each specific parameter. \\ \\
R29)If the request is allowed, the System will provide the Third Party with periodic updates about the group. \\ \\ 

\textbf{[G3.6] Allow a Third Party to visualize the available data through useful statistics.} \\ \\
R30)The System provides the Third Party with different graphical representations of the available data. \\ \\

\textbf{AutomatedSOS:} \\ \\ 
\textbf{[G4.1] NHS is alerted when the User gets in a critical state.} \\ \\
R31)The System continuously keep track of the vital parameters of the User. \\ \\
R32) The System communicate to NHS the health status of a User when his/her parameters are below certain thresholds. \\ \\
R33) The System calls the emergency number, communicates the vital parameters and the actual location of the User, and asks the dispatch of an ambulance. \\ \\

\textbf{[G4.2]Allow the User to create a list of contacts to be alerted in case of emergency.} \\ \\
R34) The system sends a message to all the numbers on the specified list in case of emergency. \\ \\
R35) The user can customize the  text message to be sent. \\ \\
R36) If there is not a customize text message, the system will send a default message. \\ \\

\textbf{Track4Run:}\\ \\
\textbf{[G5.1] Organizators can create a new run.} \\ \\
R37) The System asks the Organizator to select a starting and finishing point for the run. \\ \\
R38) The Organizator can add intermediate points that will be part of the path. \\ \\
R39) The System relies on an external service to provide the interaction with a map. \\ \\
R40) The System calculates the shortest path that includes all the points selected by the Organizator. \\ \\   
R41) The System asks the Organizator to specify the maximum number of participants. \\ \\
R42) After being created, the run is added to the list of the scheduled runs. \\ \\

\textbf{[G5.2] Allow a User to enroll to a run as participant.} \\ \\
R43) The System allow to enroll to a run if there are available entries. \\ \\
R44)The System sends an email to the User’s email address with the information related to the run and the ticket (with the bib number). \\ \\
R45) The User is notified through email updates if there are changes until the day of the run. \\ \\

\textbf{[G5.3]Allow a User to visualize the list of scheduled runs.} \\ \\
R46) The System shows only future scheduled runs. \\ \\ 
R47) A manual search can be performed by the User specifying the location range and the minimum/maximum distance. \\ \\ 
R48) The System provides the list of scheduled runs ordered by time and User’s location. \\ \\

\textbf{[G5.4] Allow a User to visualize the list of live runs.} \\ \\
R49) The list of live runs shows only the list of runs that are taking place in that moment or that are already started. \\ \\
R50) The User can decide to follow a live run from the list, directly from their device. \\ \\

\textbf{[G5.5] Allow a User to visualize on a map the positions of the participants in a live run.} \\ \\	
R51) The positions on the map are updated in real time. \\ \\
R52) The System shows the number of online visitors that are watching the run. \\


