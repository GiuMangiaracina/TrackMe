\section{Product Perspective}
\subsection{The world and the Machine}
\begin{figure}[H]
    \centering
    \includegraphics[scale=0.3]{./Pictures/worldmachine.png}

\end{figure}

Some clarifications about the diagram: \\ \\
The \underline{User actions} refer to all the behaviours the user performs in the every day life. The \underline{Third Party services} are all the services that will arise from the utilization and the study of the collected data. The \underline{Run preparation duties} are the efforts spent planning and organizing a run event projected in the real world (e.g. block the circulation of vehicles in certain streets). The \underline{Emergency ward alert} represents the background service that the machine performs to check the health status of the users with AutomatedSOS enabled. If an anomalous state is detected, the Emergency procedure is triggered.

\newpage
\subsection{UML: Class Diagram}
This is the UML model of the whole system, based on a class diagram:
\begin{figure}[H]
    \centering
    \includegraphics[scale=0.2]{Pictures/UML.png}

\end{figure}

\newpage
\subsection{State charts:}

\begin{figure}[H]
    \centering
    \includegraphics[scale=0.4]{Pictures/stateChart1.png}
    \caption{State chart of \emph{Data4Help} System}
\end{figure}

This is a background service that runs in order to keep track of the health status of each user, and notify the NHS in case of Emergency.
\begin{figure}[H]
    \centering
    \includegraphics[scale=0.4]{Pictures/stateChart2.png}
    \caption{State chart of \emph{AutomatedSOS} System}
\end{figure}
\begin{figure}[H]
    \centering
    \includegraphics[scale=0.4]{Pictures/statechart3.png}
    \caption{State chart  of \emph{Track4Run} System}
\end{figure}
