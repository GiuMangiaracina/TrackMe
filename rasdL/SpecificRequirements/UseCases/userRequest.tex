 \subsection{Request of selected data of a specific user with subscription, and data visualization}
 
The third part can exploit the Data4Help System, sending a request for the selected data (health data and location) of a specific user, either the last available data, or in subscription mode.  
 In the latter case  it inserts the interval time, i.e. how often wants to receive the updates of these data.
The user can manage his requests, and decide to accept or deny them. If the request is authorized, the third party receives the requested data from the system, according to the chosen modalities.

\begin{table}[H]
	\centering
    
    \begin{tabular}{|p{3.5cm}|p{10.3cm}|}
    
    \hline
    \textbf{\large{Actors}}  			& \tabitem Third party;\tabitem  User  									\\
    				 			
    \hline
    \textbf{\large{Goals}} 				&\ref{goal:user1}; \ref{goal:parties1};\ref{goal:parties2};\ref{goal:parties3}\\
    
    \hline
    \textbf{\large{Enter Condition}} & The third party should be logged in the Data4Help System	\\
    
    \hline
    \textbf{\large{Events Flow}}		& \begin{enumerate}[leftmargin=0.5cm]
                                          	\item The \emph{Third party}  presses the " New Request" button
                                            \item The \emph{Third party} inserts the specific fiscal code of the target user
                                            \item The \emph{Third party} specifies the parameters that want to receive
                                            
                                            \item The \emph{Third party} can select the subscription mode. If 
                            it is chosen, the third party inserts the interval time. 
                            
                            \item The \emph{System} sends an authorization request to the target user with all the related informations
                                            
                                            \item  The  \emph{user} read the request and accepts it
                                            \item The \emph{System} retrieves the data requested and sends them as soon as available to the third party. If the third party has performed a subscription request, it keeps on to send the updated data to the third party how often it has requested.
                                            \item The requested data are shown to the third party
                                
                                          \end{enumerate}
    										\\
    \hline
    \textbf{\large{Exit Condition}} 	& The third party can examine the requested data. \\
    
    \hline
    \textbf{\large{Exception}} 			& -The \emph{target user} denies the request: the negative response is communicated to the third party\newline -The selected fiscal code does not correspond to a registered user:the System asks to insert a validate one.\\
    
    \hline
    
    
    \end{tabular}
	
\end{table}

\begin{figure}[H]
    \centering
    \includegraphics[scale=0.4]{rasdL/Pictures/request1.png}
  
\end{figure}