  \subsection{Registration and Login System}
Every User who wants to use the application must be registered to the System.
After that, the User/Third Party has to log in into the System in order to be recognized and use the mobile/web application.

\begin{table}[H]
	\centering
    
    \begin{tabular}{|p{3.5cm}|p{10.3cm}|}
    
    \hline
    \textbf{\large{Actors}}  			& \tabitem Visitor\\
    				 					& \tabitem Registered User\\
                     					& \tabitem Logged-In User\\
    \hline
    \textbf{\large{Goals}} 				& \ref{goal:trackme1}                                                     \ref{goal:trackme2}\\
    
    \hline
    \textbf{\large{Enter Condition}}	& There is no enter condition for this Use Case		\\
    
    \hline
    \textbf{\large{Events Flow}}		& \begin{enumerate}[leftmargin=0.5cm]
                                          	\item The \emph{Visitor}  accesses to the web site or application log in page.
                                            \item The \emph{Visitor} inserts all the mandatory informations (username that will identify the user and password, personal data, email).
                                            \item The System registers the new User and sends back a confirmation email to the provided email address.
                                            \item The \emph{Visitor} becomes a \emph{Registered user} inserting a unique username and the password in the log in page.  
                                            \item The \emph{Registered User} now can access the \emph{TrackMe} services through the log in page.
                                            \item After the insertion of the identifier and password, the \emph{Registered User} becomes a \emph{Logged-in User}.
                                          \end{enumerate}
    										\\
    \hline
    \textbf{\large{Exit Condition}} 	& The User/Third Party is registered in the \emph{TrackMe} System, and his account is added to they system. Now the User/Third Party is able to use all the functionalities provided by the system. \\
    
    \hline
    \textbf{\large{Exception}} 			& \begin{enumerate}[leftmargin=0.5cm]
                                          	\item The \emph{Visitor} cannot register himself because is already registered.
                                          	\item The \emph{Registered User} is not able to sign in the System because the login data are wrong or he did not confirmed the confirmation email.
                                            \end{enumerate}
    										If one of these problems occur, the system both on the web site and on the application shows a message error to the User/Third Party, which is invited to re-insert their credentials or confirm the registration email.\\
    
    \hline
    
    \end{tabular}
	
\end{table}
\begin{figure}[H]

    \centering
    \includegraphics[scale=0.4]{rasdL/Pictures/login1.png}
    \caption{Sequence diagram for the registration and login process}
    
\end{figure}

