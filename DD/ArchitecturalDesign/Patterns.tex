\section{Selected architectural styles and patterns}

\begin{itemize}
    \item Layered Architecture:\\ \\
    As described in the High-level overview section, the architecture of the system is composed of four tiers: the Presentation level, the Web level, the Business Logic level and the Data level. Client executables for iOS and Android constitute the Presentation level, which will interface with the logic of the system through an Apache Web server. The database server and the external services form the Data level, which will communicate with the server application through a firewall. A layered structure improves reuse and maintainability of the system.
    \item Publish-Subscribe: \\ \\
    The subscription requests follow the event-based paradigm (often called publish-subscribe): with this modality, a Third Party subscribes to the data of a User, or of a set of Users (group), and is notified with the new data following the specified update frequency, if they are available.
    \item Observer: \\ \\
    The AutomatedSOS service can be seen as an Observer to the User health status (Observables): it continuously monitors them and activates the emergency procedures when an anomaly is detected.
    \item Client-Server: \\ \\
    The main functionalities of the system are based on the well-known client-server structure: there are two kinds of clients (Web application and Mobile application) that send requests to a unique server, which forwards them to the Business logic tier and replies to the clients.
    \item Facade: \\ \\
    All the complexity of the system is hidden by providing simple interfaces to the clients. This is evident in the AutomatedSOS Service, where all the tasks related to the management of an emergency are responsibility of the appropriate module, and the User does not need to interact directly with them.
\end{itemize}