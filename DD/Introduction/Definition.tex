\section{Definitions, Acronyms, Abbreviations}
In this part of the DD Document there are some definitions, acronyms and abbreviations that will be used among the following chapters.

\subsection{Definitions}
\begin{itemize}

\item \textbf{User:} When referring to the \emph{User}, we refer to a logged-in User which is not a Third Party. The software is usable only by a recognized User. 

\item \textbf{Visitor:} When referring to the \emph{Visitor}, we refer to a person not yet registered to the TrackMe system.

\item \textbf{Registered User:} A User registered to the TrackMe System. To use the System and performs any action, all actors must be registered. 

\item\textbf{Logged-in User:} Is a registered User who is logged in the TrackMe System.

\item\textbf{Organizer:} Is a logged-in User who uses the Track4Run system, and decides to create a new run.

\item \textbf{System:} when referring to the \emph{System}, this document refers to the entire system infrastructure.

\item \textbf{Client:} We refer to a normal logged-in User of the application, not a Third Party.

\item\textbf{Normal User:} is a Client.

\item\textbf{TrackMe:} is the name of our System, that includes the three main services: Data4Help, AutomatedSOS and Track4Run.

\item\textbf{App:} We refer to either TrackMe Mobile application or the Web version.

\item\textbf{Health data:} With health data we refer to the health data collected by User's device, among those available or requested: heart rate, blood pressure, body temperature, step counter, respiratory rate, localization.

\item\textbf{Request:} A request is performed by a Third Party and concerns one or more Users. There are 4 types of requests: One-shot User request (concerning the latest available data of a specific User selected inserting his/her fiscal code ), subscription User request (it requires an interval time to specify: the Third Party will receive the updates how often it has requested), one-shot group request (the Third Party selects the research preferences and the requested data that match them are anonymized) and subscription group request (the same of one-shot group request, but it requires to specify the update interval time). In all kind of requests the Third Party can specify which data it want to receive, among all the available health data (heart rate, blood pressure, body temperature, step counter, respiratory rate, localization). The User requests must be accepted by the User to be performed, while the group requests are performed only if concern a group of individuals greater than 1000 for a security factor.

\item\textbf{Run:} A Run is a sports competition organized by and Organizer through the Track4Run System. The runs are divided in: scheduled runs, that have been planned and will take place in the future, and live runs, that are taking place in this moment and can be followed live. The scheduled and live runs are shown in the scheduled run list and live run list respectively.

\item\textbf{Path :} Every run has a path, i.e. the route that links the points selected by the run Organizer during the event creation: initial point, the intermediate points and the final point. 

	\end{itemize}
	
\subsection{Acronyms}

\begin{itemize}
  \item \textbf{R.A.S.D.:} Requirements Analysis and Specifications Document
  \item \textbf{D.D.:} Design Document
  \item \textbf{A.P.I.:} Application Programming Interface 
  \item \textbf{N.H.S.:} National Health Service
  \item \textbf{S.M.S.:} Short Message Service
  \item \textbf{E.R.:} Entity Relationship
  \item \textbf{U.X.:} User Experience 
  \item \textbf{REST:} Representational State Transfer
  \item \textbf{DB:} Database
  \item \textbf{HTTP:} Hyper Text Transfer Protocol
  \item \textbf{HTTPS:} HTTP over SSL
\end{itemize}

\subsection{Abbreviations}
These abbreviations will be used both in this document and in the following documents.

\begin{itemize}
	\item {[}G k{]}: The k-th Goal
    \item {[}D k{]}: The k-th Domain Assumption
    \item {[}R k{]}: The k-th Functional Requirement
\end{itemize}