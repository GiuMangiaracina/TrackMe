\chapter{Implementation, Integration and Test Plan}
In this section will be expose the way in which the components and modules of the system should be developed, implemented and integrated each other, to perform an efficient testing phase.
\section{Implementation}

The development of the TrackMe System requires a first initialization phase, in which will be deployed and configured the physical servers, the security measures and the protocols which allows the component to communicate within the internal net and to the external one.
Then, will be deployed the database Server (which plays a key role within our system, since it communicates with all the modules),  accompanied by the DBMS, ready and configured, and it will proceed with the definition of the database , according to the ER structure stated in chapter 2 (so the Data layer should be done). 

At this point can begin the real implementation phase, concerning the development of the business logic tier, inside the application server.

The implementation phase should develop two software-based application:
the first is a web application, used by Third Parties, and the second one is a mobile application, used by users; thus, the two sub-projects can be entrusted to two different development teams that interacts each other and work in parallel, each with specific programming skills.
Since Data4Help data are generated by users, the development of the mobile application should begin as soon as possible, given that the Web Application bases its work on those data. In this way possible problems will be find out first.
The first module that will be develop in the user mobile service sub-system should be the User data module, because it offers the main function of the entire system, allowing the acquisition of user's data (health and location data) through user smartwatch's sensors and their storage in the database.
After that, The Web application's team can start to develop the request Module of the Third Party Web service Sub-system, while the other team will focus on  the manage request module, since these two modules are dependent.
The interaction with the Maps API is implemented in this phase, within each component development.
After the completion of these parts, the two teams can work in parallel, proceeding with the development of the remaining modules; in particular, the AutomatedSOS Module and The track4Run Module are independent, so can be develop by two different groups within the mobile application team.
The AutomatedSOS Module must be connected to the User Data Module since it is dependent to it; here are performed the connection of the module with the dial service and to the SMS gateway.


\subsection{Integration and Test Plan}
The strategy chosen for the integration plan is the bottom-up one.